%\documentclass[12pt]{scrreprt}
\documentclass[12pt]{report} 

% language may be romanian or english (default is english)
% type may be bachelor or master (default is bachelor)
\usepackage[language=english, type=bachelor]{style}

%\geometry{a4paper,top=2.5cm,left=3cm,right=2.5cm,bottom=2.5cm}
%in style
%controlling the appearance of your headers and footers
\usepackage{fancyhdr}
\pagestyle{fancy}
\lhead{}
\chead{}
\renewcommand{\headrulewidth}{0.2pt}
\renewcommand{\footrulewidth}{0.2pt}

\begin{document}

\specialization{COMPUTER SCIENCE ROMANIAN}	
\title{Depression signs detection}					   
\author{Dinica Mircea}											
\supervisor{Lector Universitar, Lupea Mihaiela}				
				
\maketitle


\newpage
\thispagestyle{empty}
\mbox{}
\newpage
\pagenumbering{roman} 

\cleardoublepage
ABSTRACT
\vspace{0.5cm}	
\hrule
\vspace{0.5cm}	
%\cleardoublepage

This scientific study delves into the realm of depression detection through the lens of artificial intelligence (AI) and natural language processing (NLP). With a focus on enhancing the accuracy and applicability of depression identification tools, the study first provides a thorough understanding of depression statistics and insights. It then outlines the primary objective: to develop a robust AI model capable of detecting depression while also assessing its performance across linguistic boundaries.

The chapters unfold to reveal a comprehensive exploration of the dataset, shedding light on its composition and characteristics. Subsequent chapters delve into the intricate process of model selection and hyperparameter tuning, aiming to optimize AI algorithms for depression detection. A pivotal aspect of the study lies in its cross-linguistic analysis, where the dataset is translated into Romanian, and the AI model is trained and evaluated on this multilingual data, offering insights into the model's performance in different linguistic contexts.

Through meticulous analysis and experimentation, this study presents valuable contributions to the field of depression detection, highlighting the importance of linguistic diversity in AI-based approaches. Ultimately, the findings pave the way for more effective and culturally inclusive depression identification tools, with implications for global mental health initiatives.



\tableofcontents


\newpage
\pagenumbering{arabic}

\chapter{Introduction}

%\chapter*{Introduction}
\label{intro}

% \par Introducere: obiectivele lucrarii si descrierea succinta a capitolelor, prezentarea temei, prezentarea contributiei proprii, respectiv a rezultatelor originale si mentionarea (daca este cazul) a sesiunii de comunicari unde a fost prezentata sau a revistei unde a fost publicata.
%THEME PRESENTATION
\section{Understanding Depression: Statistics and Insights}
\label{sec:ch1sec1}

\par \quad Depression stands as a prevalent mental health affliction with profound impacts on both psychological and physical well-being. Characterized by a disinterest in routine activities, sleep disturbances, anhedonia, and in severe cases, suicidal ideation \cite{cui2015systematic}, it has become a pervasive chronic ailment across global societies, disrupting functionality, engendering despondency, and diminishing life quality. Furthermore, individuals grappling with major depressive disorder face heightened susceptibility to cardiovascular ailments, suboptimal treatment outcomes, and elevated rates of morbidity and mortality \cite{seligman2015interface,luo2018effects}.

The World Health Organization (WHO) identifies depression as the primary contributor to global disability, affecting over 300 million individuals worldwide \cite{smith2014world}. Particularly alarming is the revelation that adolescents with severe depression are 30 times more prone to suicide \cite{stringaris2017depression}. Despite its recognized significance as a global health challenge, the intricate mechanisms underlying depression's etiology remain inadequately elucidated, albeit cultural, psychological, and biological factors are acknowledged as contributors \cite{gross2014silver,menard2016pathogenesis}.

The Global Burden of Disease (GBD) study \cite{liu2020changes} offers comprehensive insights into various ailments across 195 countries, including depression. Divided into dysthymia and major depressive disorder categories, the GBD database from 1990 to 2017 furnishes valuable data for understanding depression's prevalence trends globally. Major depressive disorder emerges as a predominant form of depression, posing a significant burden on global health, with projections indicating it may become the leading cause of disability by 2030. Moreover, while dysthymia rates decreased in some regions, it remains a concern, particularly in the United States.

Identifying underlying causes and risk factors for depression, including genetic predisposition, demographic factors, unhealthy lifestyles, and comorbidities such as stroke, cancer, and AIDS, underscores the need for multifaceted interventions and targeted policies. Governments in countries with high depression rates are urged to prioritize research, promote healthy lifestyles, and ensure comprehensive care for individuals with predisposing conditions. However, the study acknowledges limitations in data analysis, advocating for future research to delve deeper into regional risk factors and guide tailored policy interventions for effective depression control globally, As seen in Figure \ref{FigGloablDepression} \cite{liu2020changes}, which evaluated the worldwide burden of depression using the estimated annual percentage change (EAPC) and age-standardized incidence rate (ASR).

\begin{figure}[htbp]
	\centering
		\includegraphics[scale=0.65]{./figures/depression-map-Liu-et-al-2020.jpg}
	\caption{Global depression statistics comparison between 1990 and 2017 \cite{liu2020changes}}
	\label{FigGloablDepression}
\end{figure}

\section{Objective: Tool for Depression Detection and Cross-Linguistic Evaluation}
\label{sec:ch1sec2}

The objective of our scientific study is to leverage machine learning techniques to develop a robust tool for identifying depression through textual analysis. By harnessing the power of natural language processing (NLP) and artificial intelligence (AI), we aim to create a fast and reliable tool capable of detecting signs of depression in text-based communications.

The primary goal of this endeavor is twofold. Firstly, we seek to provide a timely and accessible means of identifying individuals who may be experiencing symptoms of depression. By analyzing the language used in written communications, such as social media posts, emails, or chat messages, our tool aims to offer an initial assessment of an individual's mental well-being. This proactive approach can facilitate early intervention and support, potentially mitigating the onset of more severe depressive symptoms and their associated consequences.

Secondly, we aim to evaluate the accuracy and efficacy of our machine learning model in a cross-linguistic context. To achieve this, we will translate our dataset from English to Romanian and assess the performance of the model on both language versions. This comparative analysis will enable us to ascertain the generalizability and robustness of our tool across different languages and cultural contexts.

By undertaking this study, we hope to contribute to the advancement of computational techniques for mental health assessment and intervention. Our ultimate aim is to provide clinicians, researchers, and individuals themselves with a valuable resource for early detection and prevention of depression, ultimately fostering improved mental well-being and quality of life.
%\addcontentsline{toc}{chapter}{Introducere}
%\addcontentsline{toc}{chapter}{Introduction}

\chapter{Exploring the Dataset: A Comprehensive Overview of the Textual Data}
\label{chap:ch1}

\par
\section{Dataset Composition and Preprocessing}

\quad The dataset at the heart of our exploration is a meticulously curated collection of textual data, specifically designed to advance research in the field of mental health classification. This data was amassed through a process of web scraping from various Subreddits, reflecting a wide spectrum of discussions and perspectives on mental health. The primary objective behind assembling this dataset is to enable a nuanced analysis of textual patterns that could indicate the presence or absence of depression in individuals based on their online discourse.

\subsection{Collection Methodology}
\quad The raw data was sourced by employing sophisticated web scraping techniques, targeting specific Subreddits known for their discussions on mental health issues. This approach ensured that the data collected was highly relevant to the research objectives, capturing a diverse range of experiences and expressions related to mental health.

\subsection{Dataset Overview}
\quad Comprising 7,650 unique entries, the dataset represents a rich tapestry of textual data. Each entry is meticulously annotated with an is\textunderscore depression label, distinguishing between texts that signify the presence of depression (labeled '1') and those that do not (labeled '0'). This labeling process was carried out with careful consideration to ensure accuracy and reliability in the classification.

\subsection{Data Cleaning and Preprocessing}
\quad Given the complexities and nuances of natural language, the raw data underwent a comprehensive cleaning process using multiple Natural Language Processing (NLP) techniques. This preprocessing phase was crucial for eliminating noise, such as irrelevant characters, web links, and non-English words, thereby refining the dataset for analysis. The cleaning process also involved normalizing the text to ensure consistency across the dataset, facilitating more effective data analysis and model training.

\subsection{Balancing the Dataset}
\quad A noteworthy aspect of the dataset is its well-balanced nature, with 3,900 entries labeled as non-depression ('0') and 3,831 entries indicating depression ('1'). This balance is instrumental in avoiding bias in the predictive modeling process, ensuring that the resulting classification model is both fair and accurate. By maintaining an almost equal distribution between the two categories, the dataset provides a solid foundation for developing robust algorithms capable of detecting signs of depression in textual data.


In summary, the dataset presents a comprehensive and balanced collection of textual data aimed at enhancing our understanding and classification of mental health states, specifically depression, through the lens of online discourse. The careful curation, cleaning, and balancing of the data underscore the rigor and thoughtfulness applied in preparing this dataset for research purposes. This foundation sets the stage for applying advanced NLP techniques and machine learning models to unravel the complexities of mental health classification based on textual analysis.

\section{Leveraging LIWC-22 for In-depth Textual Analysis}

\quad In the realm of textual data analysis, especially within the context of psychological research, the tool we choose to process and interpret the data is as critical as the data itself. For this reason, our exploration of the dataset employs the latest version of a highly acclaimed text analysis software, LIWC-22 (Linguistic Inquiry and Word Count). This tool represents the culmination of decades of research and development in the field of computational linguistics and psychology, designed to uncover the intricate ways in which language reflects underlying psychological states.

\quad LIWC-22 stands on the shoulders of giants, tracing its intellectual heritage back to early pioneers who first posited that the words we use in everyday communication are windows into our inner lives—revealing our thoughts, feelings, social relationships, and even our personalities. The tool is the product of a concerted effort to harness the power of computational methods to analyze language systematically, overcoming the complexities that early computer-based text analysis methods encountered \cite{boyd2022development}.

\quad With LIWC-22, researchers have at their disposal a sophisticated software tool that not only builds upon the foundation laid by previous versions but also incorporates the latest advances in text analysis. Its expanded dictionary and enhanced software capabilities make it possible to analyze language samples with unprecedented depth and precision. Whether one is interested in exploring the nuances of emotional expression, social connectivity, cognitive processes, or any other psychological dimension manifest in text, LIWC-22 offers a robust and flexible platform for investigation.

\quad In this section, we will explore the specific features of LIWC-22 that make it an invaluable tool for our research purposes, including its methodological underpinnings, its psychometric properties, and the ways in which it allows us to parse the subtle linguistic cues that signal varying psychological states. Through this exploration, readers will gain insight into the sophisticated interplay between language and psychology that LIWC-22 helps to elucidate, setting the stage for a deeper understanding of the dataset and the insights it holds.

\subsection{The Processing Capabilities of LIWC-22}

\quad The Linguistic Inquiry and Word Count (LIWC-22) tool stands as a cutting-edge solution for processing and analyzing textual data within the domain of psychosocial research. This sophisticated software, coupled with its comprehensive dictionary, bridges the gap between linguistic constructs and psychological theories, offering unparalleled insights into the psychosocial dimensions of language. Through a detailed overview of its primary and companion processing modules, we delve into how LIWC-22 serves as an indispensable tool for researchers aiming to uncover the psychological underpinnings of text \cite{boyd2022development}.

Upon analyzing texts, LIWC-22 quantitatively evaluates the language used against its expansive dictionary, calculating the percentage of words within each text that align with specific psychosocial categories. This process yields detailed metrics on the linguistic dimensions of the analyzed texts, which can be exported in various formats for further analysis.

Beyond its core functionality, LIWC-22 introduces several companion processing modules that enhance its analytical capabilities:

\begin{itemize}
\item Dictionary Workbench: Simplifying the creation of custom dictionaries, this module offers a user-friendly interface with built-in error checking. It facilitates the evaluation of custom dictionaries' psychometric properties, ensuring their effectiveness in research contexts \cite{boyd2022development}.

\item Word Frequencies and Word Clouds: These features assist in identifying the most common words within a dataset, providing visual word clouds for intuitive analysis of text samples \cite{boyd2022development}.

\item Topic Modeling with the Meaning Extraction Method: LIWC-22 incorporates the Meaning Extraction Method (MEM) for topic modeling, enabling researchers to uncover dominant themes and meanings within their datasets through factor analysis \cite{boyd2022development}.

\item Narrative Arc: This innovative module evaluates texts for narrative structures, offering insights into storytelling elements such as staging, plot progression, and cognitive tension \cite{boyd2022development}.

\item Language Style Matching (LSM): LSM analyzes the stylistic similarities between texts, offering metrics for comparing language use in various contexts, from individual communications to group dynamics \cite{boyd2022development}.

\item Contextualizer: Understanding the context of word use is vital. This module extracts words along with their surrounding text, allowing for a deeper examination of linguistic usage and implications \cite{boyd2022development}.

\item Case Studies: Tailored for in-depth analysis of individual texts, this module aggregates LIWC-22's capabilities to facilitate comprehensive study of specific documents or transcripts \cite{boyd2022development}.

\item Prepare Transcripts: Aiding in the preparation of conversation transcripts for analysis, this module streamlines the cleaning process, ensuring texts are optimized for LIWC analysis \cite{boyd2022development}.
\end{itemize}

Together, these modules position LIWC-22 as a versatile and powerful tool for linguistic and psychological research, offering novel ways to explore and interpret the complex interplay between language and psychosocial processes.

\subsection{The Evolution and Architecture of the LIWC-22 Dictionary}

\quad The LIWC-22 Dictionary is the linchpin of the Linguistic Inquiry and Word Count (LIWC) system, embodying the fusion of linguistic constructs with psychosocial theories through an extensive lexicon. This core component, comprising over 12,000 words, word stems, phrases, and select emoticons, is meticulously organized into categories and subcategories designed to capture a wide array of psychosocial constructs. This arrangement allows for a nuanced analysis of text, offering insights into the psychological state, social relationships, and cognitive processes of individuals based on their word usage.

Central to the LIWC-22 Dictionary's design is its hierarchical organization, where words are not only categorized but also interlinked across multiple dimensions. For instance, the word "cried" contributes to categories such as emotion, sadness, and past focus, illustrating the dictionary's complexity and depth. This structure enables LIWC-22 to provide a comprehensive analysis of text, reflecting various emotional and cognitive dimensions \cite{boyd2022development}.

The development of the LIWC-22 Dictionary represents a significant evolution from its predecessors, incorporating advances in computational linguistics and psychological research. The creation process involved multiple phases:
\begin{itemize}
\item Word Collection: Leveraging the foundation of the LIWC2015 dictionary, new words were generated for each category through a combination of expert input and comprehensive literature review \cite{boyd2022development}.
\item Judge Rating Phase: Words were qualitatively assessed by a panel of judges for their fit within each category, with disagreements resolved through in-depth analysis and consensus.
Base Rate Analyses: Utilizing the Meaning Extraction Helper (MEH) tool, the frequency of dictionary words in a diverse corpus was evaluated to ensure relevance and applicability across various text samples \cite{boyd2022development}.
\item Candidate Word List Generation: Through statistical analysis and expert review, candidate words were identified for potential inclusion in the dictionary, ensuring a broad and relevant lexicon \cite{boyd2022development}.
\item Psychometric Evaluation: Each category underwent rigorous testing for internal consistency, with adjustments made to optimize the dictionary's psychometric properties \cite{boyd2022development}.
\item Refinement Phase: The entire process was iteratively refined to address any oversights and enhance the dictionary's accuracy and reliability \cite{boyd2022development}.
\item Addition of Summary Variables: New summary variables were introduced to provide additional analytical dimensions, based on cutting-edge research \cite{boyd2022development}.
\end{itemize}

The LIWC-22 Dictionary has been significantly expanded to include not only traditional words but also numbers, punctuation, short phrases, and regular expressions. This expansion allows for the analysis of modern, informal communication styles found on social media and text messaging, incorporating "netspeak" and emoticons for a more comprehensive understanding of digital communication.

The dictionary's evolution reflects a balance between expert human judgment and sophisticated computational models, ensuring that LIWC-22 remains at the forefront of text analysis technology. With each iteration, LIWC has adapted to the changing landscape of language use, incorporating new categories and adjusting existing ones to better capture the psychological significance of language \cite{boyd2022development}.

In summary, the LIWC-22 Dictionary's development and structure are a testament to the interdisciplinary collaboration between linguistics and psychology. Its comprehensive and adaptable design makes it an invaluable tool for researchers and practitioners seeking to understand the deep psychosocial underpinnings of language use.

\subsection{The Psychometric Rigor of LIWC-22: Establishing Reliability and Validity}

\quad The development of the Linguistic Inquiry and Word Count (LIWC-22) tool has consistently prioritized the establishment of a scientifically robust system, focusing on both reliability and validity. This commitment has guided each iteration of LIWC, with the aim of adapting to the dynamic nature of language use and leveraging the expanding horizons of text-based data science. LIWC-22 represents the culmination of these efforts, integrating modernized dictionaries with cutting-edge data analytics to offer a highly validated tool for text analysis.

At the heart of LIWC-22's psychometric establishment is the "Test Kitchen" corpus [Figure \ref{FigKitchenCorpus}], a meticulously curated collection of English language samples drawn from a wide spectrum of sources. This corpus serves dual purposes: it is instrumental in the selection of words for the LIWC-22 dictionary and plays a crucial role in assessing the dictionary's reliability and validity. The breadth and diversity of the Test Kitchen corpus [Figure \ref{FigKitchenCorpus}] ensure that LIWC-22's analyses are grounded in a realistic representation of language use across various contexts \cite{boyd2022development}.

To capture the multifaceted nature of language, the Test Kitchen corpus [Figure \ref{FigKitchenCorpus}] was assembled from 15 distinct English language data sets, encompassing a wide range of communication forms, from blogs and emails to social media posts and movie dialogues. This comprehensive corpus consists of 15,000 texts, with each text sample reflecting the unique linguistic style of its author(s). The selection process for these samples was designed to include a diverse representation of texts, ensuring a broad coverage of language use in daily life.

The construction of this corpus involved selecting 1,000 text samples from each of the 15 sources, with each text containing at least 100 words. For longer texts, a specific algorithm was employed to extract a continuous segment of 10,000 words, ensuring a manageable and consistent analysis size. In total, the Test Kitchen corpus [Figure \ref{FigKitchenCorpus}] encompasses over 31 million words, providing a robust foundation for the validation and refinement of the LIWC-22 dictionary \cite{boyd2022development}.

Given the sensitivity and proprietary nature of some of the data sources, the Test Kitchen corpus, while invaluable for the development and testing of LIWC-22, cannot be made publicly available. This restriction underscores the careful consideration given to privacy and ethical research practices in the compilation and use of the corpus. Nevertheless, the corpus's diverse and extensive dataset has been crucial in fine-tuning LIWC-22's dictionaries to reflect genuine language usage patterns.

The meticulous construction of the Test Kitchen corpus and its application in developing LIWC-22 illustrate the comprehensive approach taken to ensure the tool's psychometric integrity. By grounding the dictionary in a wide-ranging and representative sample of English language use, LIWC-22 stands as a testament to the evolving field of text analysis, offering researchers a reliable and valid instrument for exploring the depths of linguistic and psychosocial phenomena \cite{boyd2022development}.

\begin{figure}[htbp]
	\centering
		\includegraphics[scale=0.65]{./figures/test-kitchen-corpus.png}
	\caption{The test Kitchen Corpus of 31 Million Words \cite{boyd2022development}}
	\label{FigKitchenCorpus}
\end{figure}

\subsection{Challenges and Methodologies in Assessing LIWC-22's Psychometrics}

\quad The process of quantifying the reliability and validity of text analysis tools like LIWC-22 presents unique challenges, diverging significantly from the conventional approaches used in psychological assessments. The inherent differences between verbal behavior and structured questionnaire responses necessitate a nuanced approach to evaluating the psychometric properties of LIWC categories.

Unlike self-report questionnaires that gauge a construct like anger through multiple, similar questions to ensure internal consistency, natural language does not conform to such repetitive patterns. In real-world communication—be it a social media post, an essay, or a conversation—individuals express a thought and then naturally progress to the next, without the redundant expression of the same idea. This characteristic of verbal expression implies that the psychometric standards applied to language-based analyses must be recalibrated to account for the unique dynamics of verbal behavior \cite{boyd2022development}.

The evaluation of LIWC-22's reliability involves an innovative adaptation to the language's non-repetitive nature. Taking the LIWC-22 Anger scale as an instance, the scale encompasses 181 words and phrases associated with anger. Theoretically, the usage of one anger-related word in a text should correlate with the usage of other anger-related words within the same text. By analyzing how each of these words is employed across a selection of texts and calculating the intercorrelations among these word usages, LIWC-22's approach to determining internal consistency emerges\cite{boyd2022development}.

Validating the numerous LIWC dimensions poses a significant and complex challenge. By their nature, LIWC's content categories appear to be directly relevant or face valid. Yet, the deeper question lies in determining the extent to which both personal and social psychological processes are mirrored in the use of language. For instance, the implications of using words related to "affiliation" at a high frequency raise questions about the user's social connections and needs. Are individuals using these words seeking more social interaction, or do they reflect a person's existing strong social ties? Additionally, it is important to consider whether the frequency of such language usage offers insights into or predictions about someone's social relationships and needs.

To compute these metrics, LIWC-22 employs two statistical methods: the Cronbach’s alpha (\textalpha) for continuous data, based on the percentage of total words, and the Kuder–Richardson Formula 20 (KR-20) for binary data \cite{kuder1937theory}, indicating the presence or absence of words. These methods yield insights into the internal consistency of the LIWC-22 categories, adapting traditional psychometric calculations to the context of language analysis \cite{boyd2022development}.

The application of Cronbach’s alpha in the context of LIWC-22 encounters a significant hurdle due to the variable base rates of word usage within language categories. This variability can lead to underestimations of reliability when using traditional methods. Conversely, the Kuder–Richardson Formula 20 offers a more accurate reflection of a category's internal consistency by accommodating the binary nature of word presence, thus providing a "truer" approximation of reliability in language analysis.

The volume of research at the intersection of text analysis and psychosocial processes is vast, with over 2,400 studies cited in 2021 alone that utilized LIWC for text analysis. Findings from these studies, including those from the developers' own laboratories, reveal correlations between the affect or emotion categories detected by LIWC in texts and the authors' self-reported feelings. These correlations, although modest, underscore the tool's capability to capture psychological dynamics to a certain extent. Higher correlations are observed when comparing judges' ratings of writing samples with LIWC scores, suggesting a somewhat consistent external validation of LIWC's analytical output\cite{boyd2022development}.

The methodologies adopted for assessing the reliability and validity of LIWC-22 underscore the tool's sophisticated approach to text analysis. By carefully navigating the intricacies of natural language and employing tailored statistical methods, LIWC-22 achieves a nuanced and psychometrically sound analysis of verbal behavior. This approach not only highlights the challenges inherent in language-based psychometrics but also showcases LIWC-22's commitment to providing reliable and valid insights into the psychological underpinnings of text.









\chapter{Model Selection and Hyperparameter Tuning}

\label{chap:ch3}

\quad In order to develop an accurate AI system for detecting depression from textual data, the choice of the right model and the fine-tuning of its parameters are as critical steps. This chapter describes the process of selecting Random Forest as the preferred model for our task. Known for its ability to handle complex datasets, Random Forest stands out as a powerful tool among machine learning algorithms.

\section{Reasoning Behind Choosing Random Forest}

In machine learning, selecting the most appropriate algorithm is very important to the success of any classifying task. This is also true in our case, depression detection, where the complexity and variability of the data demand an approach that is not only accurate but also can work on texts from different cultures. Looking at a  comprehensive study that evaluated twelve distinct machine learning algorithms across seven datasets\cite{siraj2023performanceModelComparison}, we made our decision to use Random Forest (RF) as the AI model.

The study \cite{siraj2023performanceModelComparison} in question compared the performance of several algorithms, including Naive Bayes (NB), Linear Discriminant Analysis (LDA), Logistic Regression (LR), Artificial Neural Networks (ANN), Support Vector Machines (SVM), K-Nearest Neighbors (K-NN), Hoeffding Tree (HT), Decision Tree (DT), C4.5, Classification and Regression Tree (CART), Random Forest (RF), and Bayesian Belief Networks (BB), across multiple metrics. Among these, Random Forest showed the most consistent and high results, showing superior accuracy, precision, and Matthew’s Correlation Coefficient (MCC). Following Random Forest, the algorithms of Neural Networks (NN), Naive Bayes (NB), Bayesian Belief Networks (BB), and Logistic Regression (LR) were identified as the next most effective, in descending order of accuracy.

The study \cite{siraj2023performanceModelComparison} also highlighted the significance of the kappa statistic and Root Mean Square Error (RMSE) as important factors in assessing model performance, further validating the consistency of Random Forest in handling diverse and complex datasets. With these statistics, and in accordance with the study’s conclusion, our selection of Random Forest is motivated by its results across multiple validation metrics.

The datasets utilized for the comparative study are varied, each with its unique characteristics and relevance to different classification tasks:
\begin{itemize}
 
\item Breast Cancer Wisconsin (Original): This dataset contains 11 attributes and is used for binary classification (two classes) with 699 instances. It does include missing values, which would require additional preprocessing steps.

\item Statlog (Vehicle Silhouettes): Comprising 19 attributes over 846 instances, this dataset is for multiclass classification with four distinct classes and has no missing values.

\item Vertebral Column: With 7 attributes and 310 instances, this dataset is also used for multiclass classification, distinguishing among three classes, without any missing values.

\item Breast Tissue: This dataset has 10 attributes across 106 instances and is used for a more complex multiclass classification task with six classes, also free of missing values.

\item Contraceptive Method Choice: It includes 10 attributes and a larger number of instances at 1473. It’s structured for multiclass classification into three classes, and there are no missing values.

\item Image Segmentation: This is a sizable dataset with 20 attributes and 2310 instances for multiclass classification involving seven classes, and it contains no missing values.

\item Artificial Characters: The largest among the datasets listed, it boasts 8 attributes across a substantial 10218 instances. It’s designed for a multiclass classification with ten classes, and like most others here, it lacks missing values.

\end{itemize}

In the context of our study focused on depression detection, our model resembles the Breast Cancer Wisconsin dataset, because we are also tackling a binary classification problem. However, our model differentiates itself with a higher dimensionality, processing 64 input attributes, which poses a greater complexity in feature representation and selection. For this dataset (Random Forest) RF achieved the highest accuracy at 97.85\%, suggesting it was the most successful in correctly identifying cases of breast cancer. It also had the highest kappa value of 95.03\%. Precision with RF was great as well, hitting a high of 98\%, while its recall was nearly as impressive at 97.9\%, showing its ability to identify most of the positive cases. 

Across the rest of the datasets analyzed in the study \cite{siraj2023performanceModelComparison}, Random Forest (RF) consistently was one of the best algorithms. Its F-measure and Matthew's Correlation Coefficient (MCC) values were notably high, often outperforming other algorithms. For instance, RF attained an accuracy of 98.48\%, kappa value of 98.23\%, and precision and recall rates both at 98.5\% on certain datasets, alongside a specificity of up to 99.7%.

While K-NN and Logistic Regression (LR) also demonstrated strong performances in certain cases, with K-NN leading in precision and recall in the Breast Tissue dataset and LR excelling with the highest MCC values for the Vehicle and Vertebral Column datasets, RF's overall dominance was clear. RF's ability to achieve the lowest error rates, coupled with the lowest root mean square error in the majority of datasets, further confirms its reliability as an algorithm for complex predictive tasks, including depression detection.
\chapter{Cross-Linguistic Analysis: Translating and Training the AI Model for Multilingual Depression Detection}
\label{chap:ch3}




\chapter{Conclusions and Future Work}
\label{conclusions}

\quad Developing a multilingual tool presents significant challenges. There is a much richer body of literature for English than for Romanian, which impacts the performance of AI models, as evidenced by our experiments. The English model achieved a precision of 96\%, significantly higher than the Romanian model's 87\%. This difference largely comes from the translation methods used and the limitations of the pre-processing tool LIWC. Despite using Google Translate, a leading translation service, the Googlelib \cite{googletranslib} encountered difficulties in maintaining the original text's meaning. Additionally, the most recent version of LIWC, LIWC-22, is only available in English, which meant a downgrade to LIWC-2015 for Romanian, which is seven years behind in advancements, as reflected in the precision of our Romanian classifier.

For future improvements, employing an AI-based translation tool could preserve text meaning more effectively. Also, since LIWC \cite{boyd2022development} specifies its use strictly for academic purposes, any commercial application would need to adopt a different pre-processing approach to cover licensing costs. Such advancements could greatly aid in targeted marketing for psychologists or enhancing mental health awareness, aligning with the goals of the "Depression Signs Detector" tool. The next step for the tool is its deployment to a production environment, transitioning from localhost to a cloud hosting platform to ensure broader accessibility and reliability.

As a computer science researcher, I cannot provide human-based validation for diagnosing depression, such validation must be performed by a specialist. Unlike object detection in images, which can be visually verified by nearly anyone, assessing the accuracy of depression detection outputs is beyond a programmer’s capability.

Moreover, while communicating, words represent only a minor fraction of the information conveyed. The tool, which solely analyzes text, is insufficient to conclusively determine if a person is depressed. Therefore, it is important to note that the tool serves merely as a preliminary assessment of a person’s emotional state, a thorough evaluation requires a professional in psychology. For a more accurate computer-based analysis, it would be necessary to consider all aspects of communication, both verbal and nonverbal.


%\addcontentsline{toc}{chapter}{Concluzii}
%\addcontentsline{toc}{chapter}{Conclusions}

\bibliography{references}

\end{document}
