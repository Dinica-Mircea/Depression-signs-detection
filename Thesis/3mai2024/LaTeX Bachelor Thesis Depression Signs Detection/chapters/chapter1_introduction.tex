\chapter{Introduction}

%\chapter*{Introduction}
\label{intro}

% \par Introducere: obiectivele lucrarii si descrierea succinta a capitolelor, prezentarea temei, prezentarea contributiei proprii, respectiv a rezultatelor originale si mentionarea (daca este cazul) a sesiunii de comunicari unde a fost prezentata sau a revistei unde a fost publicata.
%THEME PRESENTATION
\section{Depression Statistics and Insights}
\label{sec:ch1sec1}

\par \quad Depression stands as a mental health affliction with profound impacts on both psychological and physical well-being. Characterized by a disinterest in routine activities, sleep disturbances, anhedonia, and in severe cases, suicidal ideation \cite{cui2015systematic}, it has become a problem across worldwide. Furthermore, individuals with major depressive disorder face a bigger risk of cardiovascular problems, not optimal treatment outcomes, and higher rates of morbidity and mortality \cite{seligman2015interface,luo2018effects}.

The World Health Organization (WHO) identifies depression as the primary contributor to global disability, affecting over 300 million individuals worldwide \cite{smith2014world}. Particularly alarming is the revelation that adolescents with severe depression are 30 times more prone to suicide \cite{stringaris2017depression}. 
Although we know depression is a big problem worldwide, we still don't fully understand what causes it. We know that cultural, psychological, and biological factors play a part, but we don't know exactly how they all fit together.\cite{gross2014silver,menard2016pathogenesis}.

The Global Burden of Disease (GBD) study \cite{liu2020changes} offers comprehensive insights into various mental problems across 195 countries, including depression. Divided into dysthymia and major depressive disorder categories, the GBD database from 1990 to 2017 furnishes valuable data for understanding depression's evolution globally. Major depressive disorder emerges as a predominant form of depression, posing a significant burden on global health, indicating it may become the leading cause of disability by 2030. Moreover, while dysthymia rates decreased in some regions, it remains a concern, particularly in the United States.

Identifying underlying causes and risk factors for depression, including genetic predisposition, demographic factors, unhealthy lifestyles, and diseases coming from it such as stroke, cancer, and AIDS, highlights the need for a change. Governments in countries with high depression rates are encouraged to prioritize research, promote healthy lifestyles, and ensure comprehensive care for individuals with predisposing conditions. However, the study acknowledges limitations in data analysis, advocating for future research of regional risk factors and guide tailored policy interventions for effective depression control globally. This can be seen in Figure \ref{FigGloablDepression} \cite{liu2020changes}, which evaluated the worldwide burden of depression using the estimated annual percentage change (EAPC) and age-standardized incidence rate (ASR).

\begin{figure}[htbp]
	\centering
		\includegraphics[scale=0.65]{./figures/depression-map-Liu-et-al-2020.jpg}
	\caption{Global depression statistics comparison between 1990 and 2017 \cite{liu2020changes}}
	\label{FigGloablDepression}
\end{figure}

\section{Objective}
\label{sec:ch1sec2}

\quad The objective of our scientific study is to leverage machine learning techniques to develop a tool for identifying depression through textual analysis. By harnessing the power of natural language processing (NLP) and artificial intelligence (AI), we aim to create a fast and reliable tool capable of detecting signs of depression in text-based communications.

We aim to achieve two things. Firstly, we seek to provide an accessible application of identifying individuals who may be experiencing symptoms of depression. By analyzing the language used in written communications, such as social media posts, emails, or chat messages, our tool aims to offer an initial assessment of an individual's mental well-being. This approach can facilitate early intervention and support, potentially preventing more severe depressive symptoms and their associated consequences.

Secondly, we aim to evaluate the performance of our machine learning model in a cross-linguistic context. To achieve this, we will translate our dataset from English to Romanian and assess the performance of the model on both language versions. This comparative analysis will enable us to tell if of our tool is accurate across different languages and cultural contexts.

With this study we hope to contribute to the advancement of computational techniques for mental health assessment and intervention. Our aim is to provide clinicians, researchers, and individuals themselves with a valuable resource for early detection and prevention of depression, ultimately encouraging improved mental well-being and quality of life.

\section{Related Work}

\quad The field of computer science dedicated to processing text data, known as Natural Language Processing (NLP), originated in the 1950s with the Georgetown Experiment \cite{hutchins2004georgetown}, which involved automatic translation from Russian to English. Numerous NLP tasks such as machine translation, document summarization, part-of-speech tagging, named-entity recognition, and sentiment analysis have been effectively addressed to date. The techniques developed have been incorporated into many cutting-edge applications. 

Concerning binary classification for depression, another explored method was the utilization of smartphone behavioral indicators \cite{opoku2021predicting}. Employing five supervised machine learning algorithms: random forest (RF), support vector machine (SVM) with a radial basis function (RBF) kernel, and XGBoost (XGB), it attained the following performance metrics: recall between 85.55\% and 92.51\%; F1 score from 92.19\% to 95.56\%; area under receiver operating characteristic curve ranging from 88.73\% to 94.00\%; Cohen's kappa between 94.69\% and 99.06\%; and accuracy scores from 86.61\% to 92.90\%, with 96.44\% to 98.14\%.

As for the training of ML models with multi-lingual data, another research detailed the outcomes of training identical models on the original IMDB movie review dataset in English, alongside three other datasets translated into German, Hindi, and Urdu using Google Translate \cite{ghafoor2021impact}. The most accurate results in English were obtained by the support vector machine, achieving 90.45\% accuracy. For the other languages, the accuracies were 90.01\% in German, 87.26\% in Urdu, and 82.30\% in Hindi. This indicates that while translation may serve as an effective method for data acquisition in some languages, it may only be moderately successful in others.


A further investigation discussed the previously mentioned issue of translating datasets for languages with less scientific research done for, focusing on email spam detection in Urdu \cite{siddique2021machine}. The dataset, originally in English, was translated using the googletrans library \cite{googletranslib}, and training was conducted with two machine learning algorithms: support vector machine (SVM) and Naive Bayes, as well as two deep learning architectures: long short-term memory networks (LSTM) and convolutional neural networks (CNN). The models demonstrated accuracies as follows: LSTM achieved 98.4\%, CNN recorded 96.2\%, Naive Bayes reached 98.0\%, and SVM obtained 97.5\%.
