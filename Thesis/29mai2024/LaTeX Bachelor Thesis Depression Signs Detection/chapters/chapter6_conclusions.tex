\chapter{Conclusions and Future Work}
\label{conclusions}

\quad Developing a multilingual tool presents significant challenges. There is a much richer body of literature for English than for Romanian, which impacts the performance of AI models, as evidenced by our experiments. The English model achieved a precision of 96\%, significantly higher than the Romanian model's 87\%. This difference largely comes from the translation methods used and the limitations of the pre-processing tool LIWC. Despite using Google Translate, a leading translation service, the Googlelib \cite{googletranslib} encountered difficulties in maintaining the original text's meaning. Additionally, the most recent version of LIWC, LIWC-22, is only available in English, which meant a downgrade to LIWC-2015 for Romanian, which is seven years behind in advancements, as reflected in the precision of our Romanian classifier.

For future improvements, employing an AI-based translation tool could preserve text meaning more effectively. Also, since LIWC \cite{boyd2022development} specifies its use strictly for academic purposes, any commercial application would need to adopt a different pre-processing approach to cover licensing costs. Such advancements could greatly aid in targeted marketing for psychologists or enhancing mental health awareness, aligning with the goals of the "Depression Signs Detector" tool. The next step for the tool is its deployment to a production environment, transitioning from localhost to a cloud hosting platform to ensure broader accessibility and reliability.

As a computer science researcher, I cannot provide human-based validation for diagnosing depression, such validation must be performed by a specialist. Unlike object detection in images, which can be visually verified by nearly anyone, assessing the accuracy of depression detection outputs is beyond a programmer’s capability.

Moreover, while communicating, words represent only a minor fraction of the information conveyed. The tool, which solely analyzes text, is insufficient to conclusively determine if a person is depressed. Therefore, it is important to note that the tool serves merely as a preliminary assessment of a person’s emotional state, a thorough evaluation requires a professional in psychology. For a more accurate computer-based analysis, it would be necessary to consider all aspects of communication, both verbal and nonverbal.

